\documentclass[11pt, letterpaper]{practical-report}

\title[]{Statement of Teaching Philosophy \& Interests}
\author{Venkata S Govindarajan}
\date{}

\begin{document}

\maketitle

At the University of Texas at Austin, I have been fortunate to have gained considerable teaching experience as part of my Ph.D in Computational Linguistics.  I served as the \emph{Principal Instructor} for the undergraduate course \textbf{Language \& Computers} ({\rmcs LINS313}) in Summer 2022, for which I designed course materials, lectures, and assignments to introduce students to the computational techniques used to process and understand natural language, and the social impact of such technologies. I have also been the TA for various courses(itemized in my CV), and given guest lectures in-person and online, where I’ve taught a wide range of topics including probability theory, language models and vector semantics. I have immensely enjoyed connecting with students through teaching and leading discussions in these classes. Crucially, they taught me that teaching is a constant process of learning and self-evaluation. This guides my overall teaching philosophy, and helps me approach every teaching session as an exercise in growth.

My teaching philosophy is guided by a desire to be ever curious, and approach it as a cooperative exercise. While I may be the one leading the lecture or discussion, I consider myself to be working \emph{with} students towards the common goal of understanding the material for the day. From my teaching experience, I have identified three principles to guide my pedagogical goals --- building a \textbf{healthy relationship} of trust and safety with my students, \textbf{preparing material at different levels of expertise}, and a focus on learning by \textbf{application and argumentation}. I elaborate on each of these principles below, before closing with my teaching interests and a list of courses I would love to teach moving forward.

\section*{Building Trust \& Mutual Respect}

I want my students to feel free to question what I teach, as well as learn that they ought to question the assumptions and motivations behind anything they learn, no matter the source. To achieve this, I try to establish a healthy relationship based on trust and safety with my students, rather than authority and respect. For example, I believe it helps to treat any questions and intuitions that students may have with respect, and to encourage them to work through the implications of their ideas together. I learned the effectiveness of this when I was in a \textit{Pragmatics} seminar with undergraduates lead by Prof. Scott Grimm at the University of Rochester. Scott was endlessly curious, and cultivated this habit among his students -- he would engage with and encourage students to follow their intuitions on language, thus building a rapport with them. I witnessed first hand how a burning interest to question and examine even the most obvious linguistic intuitions could be infectious, and \textbf{build a culture of critical inquiry in class}. I strive to achieve this natural empathy with my students in any course that I teach. 

When teaching {\rmcs LINS313}, I found this immensely useful when giving lectures on technical concepts like Naive Bayes or language modeling. Once students were familiar with the problem we were trying to solve and \emph{why} we wanted to solve it, they could naturally think of solutions that aligned with the actual algorithm or concept at a high level. By engaging with their natural curiosity and intuitions, I found that their natural problem-seeking abilities and confidence in class flourished.

\section*{Levels of Understanding}

Students have individual needs and desires in a course which need to be balanced with the realities of a semester-long course. One way I approach this is to be prepared to \textbf{teach the material at three different levels of expertise} --- a technique I learnt under Prof. John Beavers at UT Austin. John was my instructor in Syntax, and a `Supervised Teaching' seminar --- a course where we discussed pedagogical approaches, navigating interactions and discussion sections with students, designing syllabi, and much more. John effectively used three levels of understanding to structure a lecture, and effectively communicate the material, and I applied the same technique to my class. At a high level, I want students to understand the motivations and history behind the material in a lecture --- what is something new they can do now that they couldn't before? Understanding the motivations should enables students to reach an intermediate level of understanding how and when to use the materials discussed. Finally, an intricate description of the mechanics of the material describes how or why something works at a low level, but this description may not serve all students well immediately. I believe starting with a higher-level (and moving onto a intermediate level) of understanding can be incredibly useful to students at all levels, since it welcomes novices to dig deeper into the material outside of class by giving them a head start, while fortifying understanding for all in the class. 

Students in my {\rmcs LINS313} class responded positively to this technique when I used it in the class on language modeling. Charting the history of the idea, from Claude Shannon's initial information-theoretic experiments to modern Large Language Models (LLMs) helped students understand why this idea has remained popular, why it was dismissed earlier, and why it has regained popularity now. They were prepared to learn the theory and statistical underpinnings and its applications in practice. Modern computing techniques and algorithms, like LLMs, are hard to conceptualize because of the high level of abstraction involved in their internals and operation. Moving through different levels of understanding, and aligning them with history, offers a natural path towards rich understanding.

\section*{Learning by Application \& Argumentation}

Another facet of my teaching philosophy is a focus on learning-by-application. I believe that the best way to learn various computational models, techniques and algorithms is to \textbf{use them to solve a problem}. Working through examples in class, where we challenge each other as we work through a problem step-by-step, prepares them to study it in greater detail in assignments. For example, In {\rmcs LINS313}, I relied on students solving real search and information extraction problems on my real research datasets to teach them regular expressions, as well as the underlying mechanics of how they worked.

In addition to lectures, {\rmcs LINS313} had discussion sections --- I used these as a means towards honing students argumentation and critical thinking skills. As a Computational Linguist, I believe our field's overarching goal is to work on language technologies that serve society. This places the field (and Computational Linguistics at large) squarely at \textbf{the intersection of technology and the liberal arts}. Who does machine learning serve? What are the trade-offs with building Large Language Models trained on large swathes of online text? What assumptions have we made building a model for hate speech detection using Naive Bayes or neural networks, and what effects does it have on different users? These were some of the difficult questions that we debated in class. The goal, which I think I achieved, was to enable students to learn that there are no black and white technological solutions to problems that fundamentally concern people and society --- what matters is careful consideration of values and tradeoffs \emph{before} the technology. 

\section*{Teaching Interests \& Future Courses}

My experience in various subfields of linguistics and computational linguistics puts me in a good position to teach a wide variety of classes at the undergraduate and graduate level, some of which I have detailed below. I would also be happy to teach in new areas, and develop new courses related to my research area.

\begin{description}
    \item[Introductory Courses] Introduction to Data Science, Statistics for Linguistics, Introduction to Linguistics, Semantics, Pragmatics
    \item[Advanced Courses] Machine Learning, Natural Language Processing/Introduction to Computational Linguistics, Semantics, Pragmatics, Artificial Intelligence, Neural Networks
    \item[Courses from my research \& work experience] Language, Bias \& Power, NLP for Cultural Analytics
\end{description}

Listed below are short course proposals for courses I would be happy to teach:

\paragraph{Introduction to Computational Linguistics}  This course gives an introduction to central problems and methods in Computational Linguistics, and includes hands-on programming exercises in Python to develop a deep understanding of various topics -- regular expressions, text classification, language models, neural networks, vector semantics, syntactic parsing, information extraction, dialogue systems and machine translation. The goal is to give students a glimpse into how mathematical and computational techniques can be used to answer questions in linguistics, and to acquaint them with manipulating textual data.

\paragraph{NLP For Cultural Analytics} Through weekly readings, discussions, and programming assignments, this course surveys techniques, tools, and skills needed to apply machine learning (particularly natural language processing) methods to applications in the humanities and social sciences, with a focus on the analysis of large digital text corpora, including social media, literature, and historical documents. Topics will include data collection and documentation, text processing and machine learning techniques, data visualization, modeling analysis, and ethical considerations.

\paragraph{Language, Bias, \& Power} Through weekly readings and discussions, this seminar course aims to introduce students to the various ways in which language plays a key role in perpetuating (and dismantling) bias and power structures, and how an understanding of the social meaning of utterances is essential. Various topics will be discussed -- slurs, stereotypes, politics and propaganda, language's role in social identity, language diversity, presuppositions, implicature and more.

\end{document}
