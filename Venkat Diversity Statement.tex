\documentclass[11pt, letterpaper]{practical-report}

\title[]{Diversity and Equity Statement}
\author{Venkata S Govindarajan}
\date{}

\begin{document}

\maketitle
\thispagestyle{empty}

As an Assistant Professor in Computer Science, I will strive to make inclusive classrooms and research environments that enable the success of all my students, and have developed my teaching and mentorship principles to align with this goal. As I have stated earlier, I believe Computational Linguistics lies at the intersection of technology and liberal arts. With this in mind, I will elaborate on two themes that exemplify how I approach diversity and inclusivity in my teaching and research.

\section*{Fostering Community}

For students, a supportive community of peers and mentors have proven to be crucial to their long-term success and presence in academia. This is especially true for groups under-represented in fields like Linguistics and Computer Science, where historical marginalization can lead to a lack of good mentorship and support. As part of the organizing committee on the Texas Linguistics Society conference for \href{http://tls.ling.utexas.edu/2021/}{2021} \& \href{http://tls.ling.utexas.edu/2022/}{2022}, and the \href{http://sites.utexas.edu/sxsemantics}{South by Semantics Workshop} for 2022--2024, I was involved in organizing social events (virtual and in-person) so that students from different departments like Linguistics, Philosophy and Computer Science could gather, talk to keynote speakers, build friendships and potentially collaborations. I have personally benefitted from the ideas and camaraderie that these events have offered, including making it more accessible for students to approach keynote speakers and professors who could be future mentors. I am proud that more than a few friendships and collaborations have been born through organizing these events, and look forward to building more community in my academic career by taking up similar roles. 

\section*{Inclusivity \& Opportunity}

Students at all levels benefit greatly from seeing the work of their peers, as well seniors from similar backgrounds highlighted. Both at SXSemantics and TLS, we ensured that we recruited researchers and speakers from marginalized communities and backgrounds less represented in Linguistics, Philosophy and NLP, with a special focus on early-career researchers. In addition to boosting the profile of early-career researchers and showcasing their work to the broader community, it offers a chance for students to see and interact with members of their communities further advanced in their academia journey. This helps students see that there is a path towards staying in academia and enriching the field with their background.


\end{document}
